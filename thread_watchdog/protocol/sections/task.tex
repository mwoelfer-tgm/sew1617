%!TEX root=../document.tex

\section{Einführung}
Schreibe ein Programm, welches ein simples Erzeuger-Verbraucher-Muster implementiert!

\subsection{Grundanforderungen}

\begin{itemize}
	\item Klasse Watchdog
	\item Zwei eigene Klassen (Consumer und Producer) erben von Thread
	\item Zumindest 2 Erzeuger und zwei Verbraucher 
	\item Die zwei Klassen sind über einen Queue verbunden
	\item Die Erzeuger ermitteln eine Zufallszahl (0 bis 254). Jede gefundene Zahl wird über die Queue an die Verbraucher geschickt
	\item Die Verbraucher geben Ihre erhaltene Zahl auf der Konsole aus
	\item Erzeuger und Verbraucher werden mittels Watchdog ordnungsgemäß beendet
	\item Erzeuger: Ausgabe der erstellten Zahl
	\item Verbraucher: Ausgabe der erhaltenen Zahl
	\item Die Queue hat eine Maximalgröße von 20 Elementen
	\item Kommentare und Sphinx-Dokumentation
	\item Kurzes Protokoll über deine Vorgangsweise, Aufwand, Resultate, Beobachtungen, Schwierigkeiten, ... Bitte sauberes Dokument erstellen! (Kopf- und Fußzeile etc.)
\end{itemize}


\subsection{Erweiterungen}
\begin{itemize}
	\item Erzeuger: Anzahl der vorhandenen Queue-Elemente nach dem Hinzufügen
	\item Verbraucher: Anzahl der vorhandenen Queue-Elemente nach dem Entfernen
	\item Erzeuger: Ausgabe auf der Konsole, falls die Queue voll ist
	\item Verbraucher: Ausgabe auf der Konsole, falls die Queue leer ist
	\item GUI: Darstellung des aktuellen Status der Queue (grafische Darstellung des Füllungsgrades)
\end{itemize}
\clearpage
